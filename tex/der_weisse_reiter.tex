\sect{Der weiße Reiter}

\chord{A}{W}er ist würdig, die \chord{D}{si}eben Siegel zu \chord{A}{öf}fnen?\\
Wer ist würdig? \chord{E}{K}annst du mir das \chord{A}{sa}gen?\\
Ich \chord{D}{w}einte, als ich \chord{A}{sa}h, dass \chord{E}{ni}emand würdig \chord{A}{w}ar.\\
Im \chord{D}{H}immel und auf der \chord{A}{E}rde war \chord{E}{ni}emand, der würdig gewesen \chord{A}{w}äre.\\

\chord{A}{W}eine nicht, denn \chord{D}{E}iner hat ge\chord{A}{si}egt!\\
Er ist würdig, \chord{E}{de}r Nachkomme \chord{A}{D}avids.\\
Er ist der \chord{D}{K}önig, Er ist der \chord{A}{L}öwe aus dem \chord{E}{St}amm von Je\chord{A}{hu}da.\\
Er ist Je\chord{D}{sc}hua, das \chord{A}{L}amm, das ge\chord{E}{op}fert wurde am Al\chord{A}{ta}r.

\begin{tabbing}
\chord{A}{D}er Himmel singt: „Hallelujah! \hspace{30px} \=   \textit{(Lobt den Herrn!)}\\
\chord{D}{E}hre sei Jeschua, \chord{A}{L}öwe von Jehuda, \chord{E}{G}ottes Lamm, Jeschua!“\\
Dein \chord{A}{V}olk ruft: „Maranatha! \>  \textit{(Unser Herr, komm!)}\\
\chord{D}{K}omme bald, Jeschua, \chord{A}{L}öwe von Jehuda, \chord{E}{G}ottes Lamm, Jeschua!“\\
\chord{A}{H}allelujah, Maranatha, \chord{D}{L}öwe von Jehuda, \chord{A}{G}ottes Lamm, Je\chord{E}{sc}hua!
\end{tabbing}

Am \chord{A}{Hi}mmelszelt er\chord{D}{sc}heint ein weißes Pferd.\\
\chord{E}{D}er auf ihm sitzt, ist treu und wahrhaf\chord{A}{ti}g,\\
\chord{D}{S}eine Augen \chord{A}{si}nd wie Feuerflammen,\\
\chord{D}{S}eine \chord{E}{S}timme wie Wasser\chord{A}{fl}uten,\\
\chord{D}{A}uf Seinem Haupt sind \chord{A}{vi}ele Kronen,\\
\chord{D}{A}us Seinem \chord{E}{M}und kommt ein scharfes \chord{A}{Sc}hwert.\\
Dies ist das Schwert \chord{D}{de}r Gerechtigkeit.\\
Die \chord{E}{H}eere des Himmels folgen Ihm \chord{A}{na}ch.\\
\chord{D}{S}ie reiten auf \chord{A}{w}eißen Pferden. \\
In \chord{D}{st}rahlend \chord{E}{w}eißen Leinenge\chord{A}{wä}ndern.\\
\chord{D}{S}eine Braut er\chord{A}{w}artet Ihn mit Freude. \\
Der \chord{D}{H}immel \chord{E}{ko}mmt auf die \chord{A}{E}rde.\\

\begin{footnotesize}
Liedtext entsprechend Offenbarung 5 und 19\\
Text und Melodie: Esther Judith Becker, 2011 (www.logospanoramasongs.de)
\end{footnotesize}