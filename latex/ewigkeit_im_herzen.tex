\sect{Ewigkeit im Herzen}

Ein Al\chord{C}{ta}r stand in Athen für den \chord{F}{un}bekannten \chord{C}{G}ott.\\
Für den, der im \chord{G}{H}immel \chord{C}{w}ohnt,\\
\chord{F}{D}er sich nicht unbe\chord{C}{ze}ugt gelassen hat,\\
Der die Welt er\chord{G}{sc}haffen \chord{C}{ha}t.\\
Durch Ihn allein \chord{F}{le}ben \chord{C}{w}ir, durch Ihn, der uns \chord{G}{A}tem \chord{C}{gi}bt.\\
Menschen erbauen \chord{G}{T}empel, doch Gott will dort nicht \chord{C}{w}ohnen.\\
Mit Gold, Silber und \chord{G}{S}teinen ist Er nicht darzu\chord{C}{st}ellen.\\
\chord{F}{E}r kennt alle \chord{C}{M}enschen \chord{G}{un}d setzt ihnen \chord{C}{G}renzen.\\
\chord{F}{In} der Zeit der \chord{C}{U}nwissenheit war Er für A\chord{G}{th}en noch unbe\chord{C}{ka}nnt.

\begin{tabbing}
Refrain: \= 	\chord{C}{G}ott ist nicht länger unbekannt,\\
	\> 	Durch \chord{G}{S}einen Sohn hat Er sich uns geoffen\chord{C}{ba}rt.\\
	\> 	Unergründlich ist Sein Werk,\\
	\> 	Er \chord{G}{ga}b uns die Ewigkeit ins \chord{C}{H}erz.
\end{tabbing} 

Aus Äthi\chord{C}{op}ien kam ein Mann \chord{F}{na}ch Jerusa\chord{C}{le}m;\\
Er war auf der Suche \chord{G}{na}ch dem \chord{C}{H}errn.\\
Er \chord{F}{la}s die Schriften \chord{C}{de}r Propheten,\\ 
Er las vom Mann der \chord{G}{S}chmer\chord{C}{ze}n,\\
Der uns're Leiden ge\chord{F}{tr}agen \chord{C}{ha}t,\\
Der uns're Schuld auf \chord{G}{si}ch geladen \chord{C}{ha}t.\\
Die Strafe lag auf \chord{G}{Ih}m zu unserem \chord{C}{F}rieden,\\
Sie begruben \chord{G}{Ih}n bei den Gottlo\chord{C}{se}n.\\
\chord{F}{D}och Er wird weiter\chord{C}{le}ben \chord{G}{un}d den Plan voll\chord{C}{en}den.\\
\chord{F}{D}och wen hat Je\chord{C}{sa}ja wohl gemeint?\\
Für den Äthi\chord{G}{op}ier war dies noch unbe\chord{C}{ka}nnt.\\

Refrain:	\chord{C}{G}ott ist nicht länger unbekannt ...\\

Eine \chord{C}{F}rau in Samarien \chord{F}{ka}m zum Brun\chord{C}{ne}n,\\
Und dort traf sie \chord{G}{un}seren \chord{C}{H}errn.\\
\chord{F}{S}ie wollte nun \chord{C}{en}dlich erfahren:\\
Wie soll man Gott an\chord{G}{be}\chord{C}{te}n?\\
Es zählt nicht die Um\chord{F}{ge}\chord{C}{bu}ng,\\
Es zählt nur die \chord{G}{H}erzenshal\chord{C}{tu}ng.\\
Der Himmlische \chord{G}{V}ater sucht solche als Seine An\chord{C}{be}ter,\\
Die in Seiner Wahrheit \chord{G}{le}ben und Seinen Geist in sich \chord{C}{tr}agen,\\
\chord{F}{D}enn Gott selbst ist \chord{C}{G}eist; \chord{G}{s}agte der Mes\chord{C}{si}as.\\

Refrain:	\chord{C}{G}ott ist nicht länger unbekannt ...\\

\begin{footnotesize}
Liedtext entsprechend Prediger 3, Apostelgeschichte 8 und 17, Johannes 4\\
Text und Melodie: Esther Judith Becker, 2011 (www.logospanoramasongs.de)
\end{footnotesize}